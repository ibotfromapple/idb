\documentclass[10pt]{article}
\usepackage{a4wide}

\usepackage{amsmath, amsfonts, amssymb, mathtools}
\usepackage[utf8]{inputenc}
\usepackage[T1]{fontenc}
\usepackage{color}
\usepackage{pgfplotstable}
\usepackage{array}
\usepackage{graphicx}
\graphicspath{ {pictures/} }
\usepackage{enumitem}

\title{Implementation of Databases Exercise 2}
\author{Ilya Kulikov 351063, Alina Shigabutdinova 351017, Oleg Chernikov 351016}

\begin{document}
  \maketitle
  \section*{Exercise 2.1}
  	Notation: d = drinker, ba = bar, be = beer.
  \begin{enumerate}[label=\arabic*.]
      \item
      TRC:
      \begin{equation*}
      \begin{split}
        \{l.d \mid l \in \text{likes} \wedge \exists s \in \text{serves}\ (s.be = l.be \wedge \exists f \in \text{frequents}\ (f.ba = s.ba \wedge f.d = l.d))\}
      \end{split}
      \end{equation*}
      DRC:
      \begin{equation*}
      \begin{split}
      	\{drinker \mid \exists bar, beer \\
        &\quad (drinker, bar) \in \text{frequents}\ \wedge\\
        &\quad (bar, beer) \in \text{serves}\ \wedge\\
		&\quad (drinker, beer) \in \text{likes}\ \}
      \end{split}
      \end{equation*}
      
      \item
      TRC:
      \begin{equation*}
      \begin{split}
        &\{f.d \mid f \in \text{frequents} \wedge \exists f' \in \text{frequents}\ \wedge \exists l \in \text{likes}\ \wedge \\
        &\quad l.be = \text{'Bitburger'} \wedge l.d = f'.d \wedge f'.d \neq f.d \wedge f'.ba = f.ba\}
      \end{split}
      \end{equation*}
      DRC:
      \begin{equation*}
      \begin{split}
      	\{drinker \mid \exists d',bar \\
        &\quad (drinker, bar) \in \text{frequents}\ \wedge\\
        &\quad (d', bar) \in \text{frequents}\ \wedge\\
		&\quad (d', \text{'Bitburger'}) \in \text{likes}\ \}
      \end{split}
      \end{equation*}
      
      \item
      TRC:
      \begin{equation*}
      \begin{split}
        \{f.d \mid f \in \text{frequents} \wedge \forall s \in \text{serves} (s.ba = f.ba \wedge \forall l \in \text{likes} \wedge l.d = f.d \wedge s.be \neq l.be))\}
      \end{split}
      \end{equation*}
      DRC:
      \begin{equation*}
      \begin{split}
      	\{drinker \mid \exists bar\ \forall beer \\
        &\quad (drinker, bar) \in \text{frequents}\ \wedge\\
        &\quad (bar, beer) \in \text{serves}\ \wedge\\
		&\quad (drinker, beer) \not\in \text{likes}\ \}
      \end{split}
      \end{equation*}
      
  \end{enumerate}

  \section*{Exercise 2.2}
  \begin{enumerate}[label=\arabic*.]
    \item
		Relational Algebra is used to internally represent queries and query evaluation
    plans because of several reasons: first of all, we can represent complicated queries
    by composing relational algebra operators with each other under some rules. Secondly,
    Relational Algebra is closed algebra under the finite relation domain, so we have
    definite result always. Finally, because of Codd's theorem, each RA expression
    could be represented using Relational calculus.
    \item
		A query language is relational complete, iff one can use this language to describe
    any query from Relational Algebra or even more. SQL is relational complete, as
    \textit{SELECT} corresponds to \textit{Projection}, \textit{WHERE} to \textit{Selection}
    and \textit{FROM} can perform \textit{Join} or \textit{Cartesian Product}. \textit{Rename}
    operator can also be used in SQL with key word \textit{AS}.
    \item
    The intersection RA operator returns the same rows from two relations with equivalent
    schemas. From the set theory we know, that intersection could be defined using
    difference (but in this case we lose commutative property of original intersection,
    so we are not sure about omittability):
    \begin{align}
      R \bowtie S = R \setminus (R \setminus S)
    \end{align}
    \item
  		The TRC as the name suggests operates with tuples, while DRC uses attributes and values.
  \end{enumerate}

  \section*{Exercise 2.3}
   \begin{enumerate}[label=\arabic*.]
      \item
      	For the pass 0 we produce:
        \begin{align}
          n = \frac{N}{B} = \frac{25000}{8} = 3125
        \end{align}
        Pass 1: 447 runs\\
        Pass 2: 75 runs\\
        Pass 3: 15 runs\\
        Pass 4: 4 runs\\
        Pass 5: 2 runs\\
        Pass 6: 1 run\\[3pt]
        So overall we produce 3669 runs
      \item
        The formula is $2*N*\lceil 1+log_{B-1} \lceil N/B \rceil \rceil$, where
        $\lceil 1+log_{B-1} \lceil N/B \rceil \rceil$ is corresponding to the number
        of passes. With $N = 25000$ and $B = 8$ we get $6$ as an answer.
      \item
      	The number of passes is $\lceil 1+log_{B-1} \lceil N/B \rceil \rceil = 2$,
        then $N/B = B-1$. This boils down to a quadratic equation, where one of the
        roots is negative and the other one is $B = \frac{1}{2}(1+\sqrt{1+4N})$. In case $N = 25000$
        and after ceiling we get $B = 159$
      \item
        Two-way merge sort requires $1 + \lceil log_2(N) \rceil$ passes, therefore
        the answer is 16. \\
        In the first pass we sort each page, so we require 25000 runs. For consequent passes we require 12500, 6250, 3125, ... runs. Overall, to sort the data entirely we have to use 50006 runs.
  \end{enumerate}

  \section*{Exercise 2.4}
   \begin{enumerate}[label=\arabic*.]
      \item

      \item

      \item

      \item

  \end{enumerate}


  \section*{Exercise 2.5}
   \begin{enumerate}[label=\arabic*.]
      \item
      	If each node is 70\% full, there are 14 pointers at each node. Then, $H = \lceil \log_{14}(50000) \rceil = 5$.
      \item
      	\begin{enumerate}[label=\alph*.]
          \item
			      \begin{align}
			        C = D*(1+3) = 4
			      \end{align}
          \item
            \begin{align}
              C = 50000 / 20 = 2500
            \end{align}
          \item
            \begin{align}
              C = 4 + 2 = 6
            \end{align}
          \item
          \begin{align}
            C = 50000 * 0.3 + 5 = 15005
          \end{align}
        \end{enumerate}

  \end{enumerate}

\end{document}
