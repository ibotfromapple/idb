\documentclass[10pt]{article}
\usepackage{a4wide}

\usepackage{amsmath, amsfonts, amssymb, mathtools}
\usepackage[utf8]{inputenc}
\usepackage[T1]{fontenc}
\usepackage{color}
\usepackage{pgfplotstable}
\usepackage{array}
\usepackage{graphicx}
\graphicspath{ {pictures/} }
\usepackage{enumitem}

\title{Implementation of Databases Exercise 4}
\author{Ilya Kulikov 351063, Alina Shigabutdinova 351017, Oleg Chernikov 351016}

\begin{document}
  \maketitle

  \section*{Exercise 4.1 (Query Optimization)}
  Here we have relations $R(A,B)$ and $S(B,C)$ such that $|R|=|S| = 1000000$ tuples.
  One page can hold 100 $R$ tuples and 10 $S$ tuples. Number of join possibilities
  $|R \bowtie S| = 10000$. Also we have 102 buffer pages $B$. Hence, $10000$ pages
  for $R$ and $100000$ for $S$ and $1000$ for $R$ index and $1000$ for $S$ index.\\
  %
  \subsection*{P1}
  Here we have BNLJ with R as the outer relation and S as the inner relation.\\
  \begin{align*}
    \text{IO} &= M + \frac{M}{B-2} N =\\
    &= P_R \frac{x}{100} + \frac{P_R}{B-2} \frac{x}{100} P_S =\\
    &= 100x + 100000x = 100100x;
  \end{align*}
  %
  \subsection*{P2}
  Here we have BNLJ with R as the outer relation and the index on S.B as the inner relation.\\
  \begin{align*}
    \text{IO} &= 10000 \frac{x}{100} + \frac{10000}{100} \frac{x}{100} 1000=\\
    &=100x + 1000x = 1100x;
  \end{align*}
  The best use case is $x \leq 10$.
  %
  \subsection*{P3}
  Here we have BNLJ with the index on S.B as the outer relation and R as the inner relation.\\
  \begin{align*}
    \text{IO} = 1000+\frac{1000}{100}10000\frac{x}{100} = 1000x + 1000;
  \end{align*}
  The best use case when $x = 10; x = 11$.
  %
  \subsection*{P4}
  Here we have Merge join with the index on S.B and the index on R.B. As the indexes
  are already sorted according to the join attribute B, no sorting is necessary.
  Each access to a tuple of R to retrieve the value for R.A costs one additional
  IO operation, but this needs to be done only for matching tuples of the join
  (i.e., the selection condition R.A < x is checked after the join).\\
  \begin{align*}
    \text{IO} = 1000 + 1000 + |R \bowtie S| = 12000;
  \end{align*}
  The best use case for $x \geq 11$.
\end{document}
